% Options for packages loaded elsewhere
\PassOptionsToPackage{unicode}{hyperref}
\PassOptionsToPackage{hyphens}{url}
%
\documentclass[italian,a4paper]{article}
\usepackage{amsmath,amssymb}
\usepackage{iftex}
\ifPDFTeX
  \usepackage[T1]{fontenc}
  \usepackage[utf8]{inputenc}
  \usepackage{textcomp} % provide euro and other symbols
\else % if luatex or xetex
  \usepackage{unicode-math} % this also loads fontspec
  \defaultfontfeatures{Scale=MatchLowercase}
  \defaultfontfeatures[\rmfamily]{Ligatures=TeX,Scale=1}
\fi
\usepackage{lmodern}
\ifPDFTeX\else
  % xetex/luatex font selection
\fi
% Use upquote if available, for straight quotes in verbatim environments
\IfFileExists{upquote.sty}{\usepackage{upquote}}{}
\IfFileExists{microtype.sty}{% use microtype if available
  \usepackage[]{microtype}
  \UseMicrotypeSet[protrusion]{basicmath} % disable protrusion for tt fonts
}{}
\makeatletter
\@ifundefined{KOMAClassName}{% if non-KOMA class
  \IfFileExists{parskip.sty}{%
    \usepackage{parskip}
  }{% else
    \setlength{\parindent}{0pt}
    \setlength{\parskip}{6pt plus 2pt minus 1pt}}
}{% if KOMA class
  \KOMAoptions{parskip=half}}
\makeatother
\usepackage{xcolor}
\usepackage{color}
\usepackage{fancyvrb}
\newcommand{\VerbBar}{|}
\newcommand{\VERB}{\Verb[commandchars=\\\{\}]}
\DefineVerbatimEnvironment{Highlighting}{Verbatim}{commandchars=\\\{\}}
% Add ',fontsize=\small' for more characters per line
\newenvironment{Shaded}{}{}
\newcommand{\AlertTok}[1]{\textcolor[rgb]{1.00,0.00,0.00}{\textbf{#1}}}
\newcommand{\AnnotationTok}[1]{\textcolor[rgb]{0.38,0.63,0.69}{\textbf{\textit{#1}}}}
\newcommand{\AttributeTok}[1]{\textcolor[rgb]{0.49,0.56,0.16}{#1}}
\newcommand{\BaseNTok}[1]{\textcolor[rgb]{0.25,0.63,0.44}{#1}}
\newcommand{\BuiltInTok}[1]{\textcolor[rgb]{0.00,0.50,0.00}{#1}}
\newcommand{\CharTok}[1]{\textcolor[rgb]{0.25,0.44,0.63}{#1}}
\newcommand{\CommentTok}[1]{\textcolor[rgb]{0.38,0.63,0.69}{\textit{#1}}}
\newcommand{\CommentVarTok}[1]{\textcolor[rgb]{0.38,0.63,0.69}{\textbf{\textit{#1}}}}
\newcommand{\ConstantTok}[1]{\textcolor[rgb]{0.53,0.00,0.00}{#1}}
\newcommand{\ControlFlowTok}[1]{\textcolor[rgb]{0.00,0.44,0.13}{\textbf{#1}}}
\newcommand{\DataTypeTok}[1]{\textcolor[rgb]{0.56,0.13,0.00}{#1}}
\newcommand{\DecValTok}[1]{\textcolor[rgb]{0.25,0.63,0.44}{#1}}
\newcommand{\DocumentationTok}[1]{\textcolor[rgb]{0.73,0.13,0.13}{\textit{#1}}}
\newcommand{\ErrorTok}[1]{\textcolor[rgb]{1.00,0.00,0.00}{\textbf{#1}}}
\newcommand{\ExtensionTok}[1]{#1}
\newcommand{\FloatTok}[1]{\textcolor[rgb]{0.25,0.63,0.44}{#1}}
\newcommand{\FunctionTok}[1]{\textcolor[rgb]{0.02,0.16,0.49}{#1}}
\newcommand{\ImportTok}[1]{\textcolor[rgb]{0.00,0.50,0.00}{\textbf{#1}}}
\newcommand{\InformationTok}[1]{\textcolor[rgb]{0.38,0.63,0.69}{\textbf{\textit{#1}}}}
\newcommand{\KeywordTok}[1]{\textcolor[rgb]{0.00,0.44,0.13}{\textbf{#1}}}
\newcommand{\NormalTok}[1]{#1}
\newcommand{\OperatorTok}[1]{\textcolor[rgb]{0.40,0.40,0.40}{#1}}
\newcommand{\OtherTok}[1]{\textcolor[rgb]{0.00,0.44,0.13}{#1}}
\newcommand{\PreprocessorTok}[1]{\textcolor[rgb]{0.74,0.48,0.00}{#1}}
\newcommand{\RegionMarkerTok}[1]{#1}
\newcommand{\SpecialCharTok}[1]{\textcolor[rgb]{0.25,0.44,0.63}{#1}}
\newcommand{\SpecialStringTok}[1]{\textcolor[rgb]{0.73,0.40,0.53}{#1}}
\newcommand{\StringTok}[1]{\textcolor[rgb]{0.25,0.44,0.63}{#1}}
\newcommand{\VariableTok}[1]{\textcolor[rgb]{0.10,0.09,0.49}{#1}}
\newcommand{\VerbatimStringTok}[1]{\textcolor[rgb]{0.25,0.44,0.63}{#1}}
\newcommand{\WarningTok}[1]{\textcolor[rgb]{0.38,0.63,0.69}{\textbf{\textit{#1}}}}
\usepackage{longtable,booktabs,array}
\usepackage{calc} % for calculating minipage widths
% Correct order of tables after \paragraph or \subparagraph
\usepackage{etoolbox}
\makeatletter
\patchcmd\longtable{\par}{\if@noskipsec\mbox{}\fi\par}{}{}
\makeatother
% Allow footnotes in longtable head/foot
\IfFileExists{footnotehyper.sty}{\usepackage{footnotehyper}}{\usepackage{footnote}}
\makesavenoteenv{longtable}
\usepackage{graphicx}
\makeatletter
\def\maxwidth{\ifdim\Gin@nat@width>\linewidth\linewidth\else\Gin@nat@width\fi}
\def\maxheight{\ifdim\Gin@nat@height>\textheight\textheight\else\Gin@nat@height\fi}
\makeatother
% Scale images if necessary, so that they will not overflow the page
% margins by default, and it is still possible to overwrite the defaults
% using explicit options in \includegraphics[width, height, ...]{}
\setkeys{Gin}{width=\maxwidth,height=\maxheight,keepaspectratio}
% Set default figure placement to htbp
\makeatletter
\def\fps@figure{htbp}
\makeatother
\usepackage{svg}
\setlength{\emergencystretch}{3em} % prevent overfull lines
\providecommand{\tightlist}{%
  \setlength{\itemsep}{0pt}\setlength{\parskip}{0pt}}
\setcounter{secnumdepth}{5}
\ifLuaTeX
  \usepackage{selnolig}  % disable illegal ligatures
\fi
\usepackage{bookmark}
\IfFileExists{xurl.sty}{\usepackage{xurl}}{} % add URL line breaks if available
\urlstyle{same}
\hypersetup{
  pdftitle={Manuale d\textquotesingle Informatica - Esame di Stato 2025 - 5A SIA - IIS Savoia Benincasa},
  hidelinks,
  pdfcreator={LaTeX via pandoc}}

\title{Manuale d\textquotesingle Informatica - Esame di Stato 2025 - 5A
SIA - IIS Savoia Benincasa}
\author{}
\date{}

\begin{document}
\maketitle

{
\setcounter{tocdepth}{6}
\tableofcontents
}
\section{Progettazione
dell'applicazione}\label{progettazione-dellapplicazione}

\subsection{Specifiche dei Requisiti - Diagramma dei casi
d'uso}\label{specifiche-dei-requisiti---diagramma-dei-casi-duso}

\includesvg{esempioUseCase.svg}

\subsection{Progettazione dei dati}\label{progettazione-dei-dati}

\subsubsection{Diagramma delle classi}\label{diagramma-delle-classi}

\paragraph{Classe}\label{classe}


\includesvg{esempioClasse.svg}

\paragraph{Associazione}\label{associazione}

\subparagraph{Associazione uno a uno}\label{associazione-uno-a-uno}

Progettazione concettuale - Classi di analisi

\includesvg{esempioAss11Conc.svg}

Letture dell'associazione:

\begin{itemize}
\tightlist
\item
  Uno studente può essere identificato da un badge
\item
  Un badge identifica uno (ed un solo) studente
\end{itemize}

Progettazione logica - Classi di progettazione

\includesvg{esempioAss11Ristr1.svg}

Ristrutturazione nel modello logico relazionale

\includesvg{esempioAss11Ristr2.svg}

Schema logico

\begin{Shaded}
\begin{Highlighting}[]
\NormalTok{Studente(matricola \textless{}PK\textgreater{}, codiceBadge, …)}
\end{Highlighting}
\end{Shaded}

DDL - SQL

\begin{Shaded}
\begin{Highlighting}[]
\KeywordTok{CREATE} \KeywordTok{TABLE}\NormalTok{ Studente (}
\NormalTok{    matricola }\DataTypeTok{INTEGER} \KeywordTok{PRIMARY} \KeywordTok{KEY}\NormalTok{,}
\NormalTok{    codiceBadge }\DataTypeTok{INTEGER}\NormalTok{,}
\NormalTok{    …}
\NormalTok{);}
\end{Highlighting}
\end{Shaded}

\subparagraph{Associazione uno a molti}\label{associazione-uno-a-molti}

Progettazione concettuale - Classi di analisi

\includesvg{esempioAss1nConc.svg}

Progettazione logica - Classi di progettazione

\includesvg{esempioAss1nRistr1.svg}

Schema logico

\begin{Shaded}
\begin{Highlighting}[]
\NormalTok{Cliente(idCliente \textless{}PK\textgreater{}, …)}
\NormalTok{Prodotto(idProdotto \textless{}PK\textgreater{}, idCliente \textless{}FK\textgreater{}, …)}
\end{Highlighting}
\end{Shaded}

DDL - SQL

\begin{Shaded}
\begin{Highlighting}[]
\KeywordTok{CREATE} \KeywordTok{TABLE}\NormalTok{ Cliente (}
\NormalTok{    idCliente }\DataTypeTok{INTEGER} \KeywordTok{PRIMARY} \KeywordTok{KEY}\NormalTok{,}
\NormalTok{    …}
\NormalTok{);}
\KeywordTok{CREATE} \KeywordTok{TABLE}\NormalTok{ Prodotto (}
\NormalTok{    idProdotto }\DataTypeTok{INTEGER} \KeywordTok{PRIMARY} \KeywordTok{KEY}\NormalTok{,}
\NormalTok{    idCliente }\DataTypeTok{INTEGER} \KeywordTok{FOREIGN} \KeywordTok{KEY} \KeywordTok{REFERENCES}\NormalTok{ Cliente(idCliente),}
\NormalTok{    …}
\NormalTok{);}
\end{Highlighting}
\end{Shaded}

\subparagraph{Associazione molti a
molti}\label{associazione-molti-a-molti}

Progettazione concettuale - Classi di analisi

\includesvg{esempioAssnnConc.svg}

Progettazione logica - Classi di progettazione

\includesvg{esempioAssnnRistr1.svg}

Ristrutturazione nel modello logico relazionale

\includesvg{esempioAssnnRistr2.svg}

Schema logico

\begin{Shaded}
\begin{Highlighting}[]
\NormalTok{Societa(partitaIVA \textless{}PK\textgreater{}, …)}
\NormalTok{Azionista(CF \textless{}PK\textgreater{}, …)}
\NormalTok{Partecipazione(partitaIVA \textless{}PK, FK\textgreater{}, CF \textless{}PK, FK\textgreater{}, quota)}
\end{Highlighting}
\end{Shaded}

DDL-SQL

\begin{Shaded}
\begin{Highlighting}[]
\KeywordTok{CREATE} \KeywordTok{TABLE}\NormalTok{ Societa (}
\NormalTok{    partitaIVA TEXT }\KeywordTok{PRIMARY} \KeywordTok{KEY} \KeywordTok{CHECK}\NormalTok{ (}\FunctionTok{length}\NormalTok{(partitaIVA) }\OperatorTok{=} \DecValTok{11}\NormalTok{),}
\NormalTok{    …}
\NormalTok{);}
\KeywordTok{CREATE} \KeywordTok{TABLE}\NormalTok{ Azionista (}
\NormalTok{    CF TEXT }\KeywordTok{PRIMARY} \KeywordTok{KEY} \KeywordTok{CHECK}\NormalTok{ (}\FunctionTok{length}\NormalTok{(CF) }\OperatorTok{=} \DecValTok{16}\NormalTok{),}
\NormalTok{    …}
\NormalTok{);}
\KeywordTok{CREATE} \KeywordTok{TABLE}\NormalTok{ Partecipazione (}
\NormalTok{    partitaIVA TEXT }\KeywordTok{REFERENCES}\NormalTok{ Societa(partitaIVA),}
\NormalTok{    CF TEXT }\KeywordTok{REFERENCES}\NormalTok{ Azionista(CF),}
    \KeywordTok{quota} \DataTypeTok{REAL}\NormalTok{,}
    \KeywordTok{PRIMARY} \KeywordTok{KEY}\NormalTok{(partitaIVA, CF)}
\NormalTok{);}
\end{Highlighting}
\end{Shaded}


\subsubsection{Ristrutturazione del modello concettuale in quello logico
relazionale}\label{ristrutturazione-del-modello-concettuale-in-quello-logico-relazionale}

\begin{enumerate}
\def\labelenumi{\arabic{enumi}.}
\tightlist
\item
  ogni \emph{entità} diventa una \emph{relazione}, ossia una tabella
  SQL;
\item
  ogni \emph{attributo} di un'entità diventa un \emph{attributo} della
  relazione, cioè il nome di una colonna della tabella SQL;
\item
  ogni \emph{attributo} della relazione eredita le caratteristiche
  dell'attributo dell'entità da cui deriva;
\item
  l'identifi catore univoco di un'entità diventa la chiave primaria
  della relazione derivata;
\item
  l'associazione uno a uno diventa un'unica relazione che contiene gli
  attributi della prima e della seconda entità, salvo alcune eccezioni;
\item
  l'associazione uno a molti viene rappresentata aggiungendo, agli
  attributi dell'entità che svolge il ruolo a molti, l'identifi catore
  univoco dell'entità che svolge il ruolo a unonell'associazione. Questo
  identificatore, che prende il nome di chiave esterna dell'entità
  associata, è costituito dall'insieme di attributi che compongono la
  chiave dell'entitàa uno dell'associazione. Gli eventuali attributi
  dell'associazione vengono inseriti nella relazione che rappresenta
  l'entità a molti, assieme alla chiave esterna;
\item
  l'associazione molti a molti diventa una nuova relazione (in aggiunta
  alle relazioni derivate dalle entità) composta dagli identifi catori
  univoci delle due entità e dagli eventuali attributi
  dell'associazione. La chiave della nuova relazione è formata
  dall'insieme di attributi che compongo le chiavi delle due entità,
  oltre agli eventuali attributi dell'associazione necessari a garantire
  l'unicità delle n-uple nella relazione ottenuta.
\end{enumerate}

\subsection{SQL}\label{sql}

\subsubsection{Tipi di dato}\label{tipi-di-dato}

\paragraph{SQLite}\label{sqlite}

\begin{itemize}
\item
  \textbf{\texttt{INTEGER}} Valore intero con segno.
\item
  \textbf{\texttt{REAL}} Valore numerico ``reale''.
\item
  \textbf{\texttt{TEXT}} Una stringa di caratteri.
\item
  \textbf{\texttt{BLOB}} (Binary Large OBject) Una rappresentazione
  binaria di un qualunque file.
\end{itemize}

\paragraph{Altri possibili tipi}\label{altri-possibili-tipi}

\begin{itemize}
\tightlist
\item
  \textbf{\texttt{BOOL}} \texttt{FALSE} o \texttt{TRUE}. In SQLite si
  usa \texttt{INTEGER} con la convenzione per cui \texttt{FALSE\ =\ 0} e
  \texttt{TRUE\ =\ 1}
\item
  \textbf{\texttt{DATE}} Conserva la data. In SQLite possiamo usare
  \texttt{TEXT} con date scritte secondo lo standard ISO 8601:
  ``YYYY-MM-DD''.
\item
  \textbf{\texttt{DATETIME}}. Conserva l'istante temporale. In SQLite
  possiamo usare \texttt{TEXT} e lo standard ISO 8601: ``YYYY-MM-DD
  HH:MM:SS.SSS''.
\end{itemize}

\subsubsection{Sintassi Base - SELECT in
SQLite}\label{sintassi-base---select-in-sqlite}

Le parentesi quadre (\texttt{{[}} e \texttt{{]}}) indicano
l'opzionalità.

\begin{Shaded}
\begin{Highlighting}[]
\KeywordTok{SELECT}\NormalTok{ colonne}
\NormalTok{[}\KeywordTok{FROM}\NormalTok{ tabella]}
\NormalTok{[}\KeywordTok{WHERE}\NormalTok{ condizione]}
\NormalTok{[}\KeywordTok{GROUP} \KeywordTok{BY}\NormalTok{ colonne\_raggruppamento]}
\NormalTok{[}\KeywordTok{HAVING}\NormalTok{ condizione\_raggruppamento]}
\NormalTok{[}\KeywordTok{ORDER} \KeywordTok{BY}\NormalTok{ colonne\_ordinamento [}\KeywordTok{ASC}\NormalTok{|}\KeywordTok{DESC}\NormalTok{]]}
\NormalTok{[}\KeywordTok{LIMIT}\NormalTok{ numero [OFFSET inizio]];}
\end{Highlighting}
\end{Shaded}

\begin{Shaded}
\begin{Highlighting}[]
\NormalTok{colonne := espressione [, espressione]*}
\NormalTok{espressione := nome\_colonna |}
\NormalTok{               letterale |}
\NormalTok{               espressione AS nome |}
\NormalTok{               espressione + espressione |}
\NormalTok{               espressione {-} espressione |}
\NormalTok{               espressione * espressione |}
\NormalTok{               espressione / espresssione |}
\NormalTok{               min(espressione) |}
\NormalTok{               max(espressione) |}
\NormalTok{               count(espressione) |}
\NormalTok{               avg(espressione) |}
\NormalTok{               sum(espressione) |}
\NormalTok{               espressione = espressione |}
\NormalTok{               espressione \textless{}\textgreater{} espressione |}
\NormalTok{               espressione \textless{}= espressione |}
\NormalTok{               espressione \textless{} espressione |}
\NormalTok{               espressione \textgreater{}= espressione |}
\NormalTok{               espressione \textgreater{} espressione}
\NormalTok{               espressione BETWEEN espressione AND espressione;}

\NormalTok{tabella := nome\_tabella |}
\NormalTok{           nome\_tabella, nome\_tabella |}
\NormalTok{           nome\_tabella join nome\_tabella clausola\_join;}

\NormalTok{join := , |}
\NormalTok{        INNER JOIN |}
\NormalTok{        CROSS JOIN |}
\NormalTok{        LEFT OUTER JOIN |}
\NormalTok{        RIGHT OUTER JOIN |}
\NormalTok{        FULL OUTER JOIN |}
\NormalTok{        NATURAL JOIN;}

\NormalTok{clausola\_join : ON condizione |}
\NormalTok{                USING(nome\_attributo) |}
\NormalTok{                "";}

\NormalTok{condizione := FALSE | TRUE |}
\NormalTok{              condizione AND condizione |}
\NormalTok{              condizione OR condizione |}
\NormalTok{              NOT condizione |}
\NormalTok{              espressione}
\end{Highlighting}
\end{Shaded}

\begin{itemize}
\tightlist
\item
  \texttt{SELECT} \emph{colonne}: Specifica le colonne che si desidera
  visualizzare nel risultato della query. È possibile specificare una o
  più colonne separate da virgole. Utilizzare \texttt{*} per selezionare
  tutte le colonne della tabella. È possibile utilizzare alias per le
  colonne usando la parola chiave \texttt{AS} (es.
  \texttt{nome\_colonna\ AS\ alias}). Si possono applicare funzioni
  aggregate (es. \texttt{COUNT()}, \texttt{SUM()}, \texttt{AVG()},
  \texttt{MIN()}, \texttt{MAX()}) alle colonne.
\item
  {[}\texttt{FROM} \emph{tabella}{]}: Indica la tabella o le tabelle da
  cui recuperare i dati. Se si interrogano più tabelle, è necessario
  specificarle separate da virgole (e solitamente utilizzare clausole
  \texttt{JOIN} ).
\item
  {[}\texttt{WHERE} \emph{condizione}{]}: Filtra le righe in base a una
  condizione specificata. La condizione può includere operatori di
  confronto (\texttt{=}, \texttt{\textgreater{}}, \texttt{\textless{}},
  \texttt{\textgreater{}=}, \texttt{\textless{}=}, \texttt{!=},
  \texttt{\textless{}\textgreater{}}), operatori logici (\texttt{AND},
  \texttt{OR}, \texttt{NOT}), operatori \texttt{IN}, \texttt{BETWEEN},
  \texttt{LIKE}, \texttt{IS\ NULL}, \texttt{IS\ NOT\ NULL}.
\item \texttt{GROUP BY} ordina per gli attributi
\item \texttt{HAVING} filtra sugli attributi aggregati
\item \texttt{ORDER BY} ordina per
  \begin{itemize}
  \tightlist
  \item
    \texttt{ASC} (ascendente) è l'ordine predefinito.
  \item
    \texttt{DESC} (discendente) ordina dal valore più alto al più basso.
  \end{itemize}
\item
  \texttt{OFFSET} \emph{inizio} (opzionale) specifica il numero di righe
  da saltare prima di iniziare a restituire i risultati.
\end{itemize}

\subsubsection{Date}\label{date}

\paragraph{ISO 8601 (Representation of dates and
times)}\label{iso-8601-representation-of-dates-and-times}

\texttt{YYYY-MM-DD}

date()

\begin{itemize}
\tightlist
\item
  Giorno:
  \texttt{substr(\textquotesingle{}2025-04-12\textquotesingle{},\ 9,\ 2)}
\item
  Mese:
  \texttt{substr(\textquotesingle{}2025-04-12\textquotesingle{},\ 6,\ 2)}
\item
  Anno:
  \texttt{substr(\textquotesingle{}2025-04-12\textquotesingle{},\ 1,\ 4)}
\item
  Mese corrente: \texttt{substr(date(),\ 6,\ 2)}
\item
  Anno corrente: \texttt{substr(date(),\ 1,\ 4)}
\end{itemize}

\subsection{Web}\label{web}

\subsubsection{HTML}\label{html}

\begin{longtable}[]{@{}
  >{\raggedright\arraybackslash}p{(\columnwidth - 2\tabcolsep) * \real{0.1809}}
  >{\raggedright\arraybackslash}p{(\columnwidth - 2\tabcolsep) * \real{0.8191}}@{}}
\toprule\noalign{}
\begin{minipage}[b]{\linewidth}\raggedright
Nome tag
\end{minipage} & \begin{minipage}[b]{\linewidth}\raggedright
Descrizione
\end{minipage} \\
\midrule\noalign{}
\endhead
\bottomrule\noalign{}
\endlastfoot
\texttt{\textless{}html\textgreater{}} & La radice di un documento HTML.
Tutti gli altri elementi sono discendenti di questo tag. \\
\texttt{\textless{}head\textgreater{}} & I metadati del documento HTML,
come il titolo, set di caratteri, link a fogli di stile, ecc. Questi non
sono visualizzati direttamente nella pagina. \\
\texttt{\textless{}title\textgreater{}} & Il titolo del documento, che
appare nella barra del titolo del browser o nella scheda della
pagina. \\
\texttt{\textless{}body\textgreater{}} & Il contenuto visibile del
documento HTML (testo, immagini, link, ecc.). \\
\texttt{\textless{}h1\textgreater{}} -
\texttt{\textless{}h6\textgreater{}} & Le intestazioni di diverso
livello (da quella più importante \texttt{\textless{}h1\textgreater{}} a
quella meno importante \texttt{\textless{}h6\textgreater{}}). \\
\texttt{\textless{}p\textgreater{}} & Un capoverso
(\emph{paragraph}). \\
\texttt{\textless{}a\textgreater{}} & Un hyperlink (collegamento).
L'attributo \texttt{href} specifica l'URL di destinazione. \\
\texttt{\textless{}img\textgreater{}} & Un'immagine nel documento.
L'attributo \texttt{src} specifica il percorso dell'immagine. \\
\texttt{\textless{}ul\textgreater{}} & Una lista non ordinata (con punti
elenco). \\
\texttt{\textless{}ol\textgreater{}} & Una lista ordinata (con numeri o
lettere). \\
\texttt{\textless{}li\textgreater{}} & Un elemento di una lista (sia
ordinata che non ordinata). \\
\texttt{\textless{}div\textgreater{}} & Una sezione o un contenitore
generico per altri elementi HTML. \\
\texttt{\textless{}span\textgreater{}} & Una sezione o un contenitore
inline generico per altri elementi HTML. Simile a
\texttt{\textless{}div\textgreater{}}, ma per elementi inline. \\
\texttt{\textless{}table\textgreater{}} & Una tabella. \\
\texttt{\textless{}tr\textgreater{}} & Una riga all'interno di una
tabella. \\
\texttt{\textless{}th\textgreater{}} & Una cella di intestazione in una
tabella. \\
\texttt{\textless{}td\textgreater{}} & Una cella di dati in una
tabella. \\
\texttt{\textless{}form\textgreater{}} & Un modulo HTML utilizzato per
raccogliere l'input dell'utente. \\
\texttt{\textless{}input\textgreater{}} & Un campo di input all'interno
di un modulo (testo, password, pulsante, ecc.). \\
\texttt{\textless{}button\textgreater{}} & Un pulsante cliccabile. \\
\texttt{\textless{}select\textgreater{}} & Un menu a tendina (lista di
opzioni). \\
\texttt{\textless{}option\textgreater{}} & Un'opzione all'interno di un
elemento \texttt{\textless{}select\textgreater{}}. \\
\texttt{\textless{}strong\textgreater{}} & Evidenzia il testo con una
forte enfasi (solitamente visualizzato in grassetto). \\
\texttt{\textless{}em\textgreater{}} & Enfatizza il testo (solitamente
visualizzato in corsivo). \\
\texttt{\textless{}br\textgreater{}} & Un'interruzione di riga
singola. \\
\texttt{\textless{}hr\textgreater{}} & Una linea orizzontale tematica
(separatore). \\
\end{longtable}

\paragraph{Pagina web vouta - HTML}\label{pagina-web-vouta---html}

\begin{Shaded}
\begin{Highlighting}[]
\DataTypeTok{\textless{}!DOCTYPE}\NormalTok{ html}\DataTypeTok{\textgreater{}}
\DataTypeTok{\textless{}}\KeywordTok{html}\OtherTok{ lang}\OperatorTok{=}\StringTok{"it"}\DataTypeTok{\textgreater{}}
\DataTypeTok{\textless{}}\KeywordTok{head}\DataTypeTok{\textgreater{}}
    \DataTypeTok{\textless{}}\KeywordTok{meta}\OtherTok{ charset}\OperatorTok{=}\StringTok{"UTF{-}8"}\DataTypeTok{\textgreater{}}
    \DataTypeTok{\textless{}}\KeywordTok{title}\DataTypeTok{\textgreater{}}\NormalTok{Pagina Vuota}\DataTypeTok{\textless{}/}\KeywordTok{title}\DataTypeTok{\textgreater{}}
\DataTypeTok{\textless{}/}\KeywordTok{head}\DataTypeTok{\textgreater{}}
\DataTypeTok{\textless{}}\KeywordTok{body}\DataTypeTok{\textgreater{}}

\DataTypeTok{\textless{}/}\KeywordTok{body}\DataTypeTok{\textgreater{}}
\DataTypeTok{\textless{}/}\KeywordTok{html}\DataTypeTok{\textgreater{}}
\end{Highlighting}
\end{Shaded}

% \paragraph{Pagina web vouta - Document Object Model}\label{pagina-web-vouta---document-object-model}

\includesvg{dom_vuoto.svg}

\paragraph{Tabella - HTML}\label{tabella---html}

\begin{Shaded}
\begin{Highlighting}[]
\DataTypeTok{\textless{}}\KeywordTok{table}\DataTypeTok{\textgreater{}}
  \DataTypeTok{\textless{}}\KeywordTok{caption}\DataTypeTok{\textgreater{}}\NormalTok{Tabella di Esempio}\DataTypeTok{\textless{}/}\KeywordTok{caption}\DataTypeTok{\textgreater{}}
  \DataTypeTok{\textless{}}\KeywordTok{thead}\DataTypeTok{\textgreater{}}
    \DataTypeTok{\textless{}}\KeywordTok{tr}\DataTypeTok{\textgreater{}}
      \DataTypeTok{\textless{}}\KeywordTok{th}\DataTypeTok{\textgreater{}}\NormalTok{Intestazione Colonna 1}\DataTypeTok{\textless{}/}\KeywordTok{th}\DataTypeTok{\textgreater{}}
      \DataTypeTok{\textless{}}\KeywordTok{th}\DataTypeTok{\textgreater{}}\NormalTok{Intestazione Colonna 2}\DataTypeTok{\textless{}/}\KeywordTok{th}\DataTypeTok{\textgreater{}}
    \DataTypeTok{\textless{}/}\KeywordTok{tr}\DataTypeTok{\textgreater{}}
  \DataTypeTok{\textless{}/}\KeywordTok{thead}\DataTypeTok{\textgreater{}}
  \DataTypeTok{\textless{}}\KeywordTok{tbody}\DataTypeTok{\textgreater{}}
    \DataTypeTok{\textless{}}\KeywordTok{tr}\DataTypeTok{\textgreater{}}
      \DataTypeTok{\textless{}}\KeywordTok{td}\DataTypeTok{\textgreater{}}\NormalTok{Dato Riga 1, Colonna 1}\DataTypeTok{\textless{}/}\KeywordTok{td}\DataTypeTok{\textgreater{}}
      \DataTypeTok{\textless{}}\KeywordTok{td}\DataTypeTok{\textgreater{}}\NormalTok{Dato Riga 1, Colonna 2}\DataTypeTok{\textless{}/}\KeywordTok{td}\DataTypeTok{\textgreater{}}
    \DataTypeTok{\textless{}/}\KeywordTok{tr}\DataTypeTok{\textgreater{}}
    \DataTypeTok{\textless{}}\KeywordTok{tr}\DataTypeTok{\textgreater{}}
      \DataTypeTok{\textless{}}\KeywordTok{td}\DataTypeTok{\textgreater{}}\NormalTok{Dato Riga 2, Colonna 1}\DataTypeTok{\textless{}/}\KeywordTok{td}\DataTypeTok{\textgreater{}}
      \DataTypeTok{\textless{}}\KeywordTok{td}\DataTypeTok{\textgreater{}}\NormalTok{Dato Riga 2, Colonna 2}\DataTypeTok{\textless{}/}\KeywordTok{td}\DataTypeTok{\textgreater{}}
    \DataTypeTok{\textless{}/}\KeywordTok{tr}\DataTypeTok{\textgreater{}}
  \DataTypeTok{\textless{}/}\KeywordTok{tbody}\DataTypeTok{\textgreater{}}
  \DataTypeTok{\textless{}}\KeywordTok{tfoot}\DataTypeTok{\textgreater{}}
    \DataTypeTok{\textless{}}\KeywordTok{tr}\DataTypeTok{\textgreater{}}
      \DataTypeTok{\textless{}}\KeywordTok{td}\OtherTok{ colspan}\OperatorTok{=}\StringTok{"2"}\DataTypeTok{\textgreater{}}\NormalTok{Nota a piè di pagina della tabella}\DataTypeTok{\textless{}/}\KeywordTok{td}\DataTypeTok{\textgreater{}}
    \DataTypeTok{\textless{}/}\KeywordTok{tr}\DataTypeTok{\textgreater{}}
  \DataTypeTok{\textless{}/}\KeywordTok{tfoot}\DataTypeTok{\textgreater{}}
\DataTypeTok{\textless{}/}\KeywordTok{table}\DataTypeTok{\textgreater{}}
\end{Highlighting}
\end{Shaded}

\begin{longtable}[]{@{}ll@{}}
\caption{Tabella di Esempio}\tabularnewline
\toprule\noalign{}
Intestazione Colonna 1 & Intestazione Colonna 2 \\
\midrule\noalign{}
\endfirsthead
\toprule\noalign{}
Intestazione Colonna 1 & Intestazione Colonna 2 \\
\midrule\noalign{}
\endhead
\midrule\noalign{}
\multicolumn{2}{@{}l@{}}{%
Nota a piè di pagina della tabella} \\
\bottomrule\noalign{}
\endlastfoot
Dato Riga 1, Colonna 1 & Dato Riga 1, Colonna 2 \\
Dato Riga 2, Colonna 1 & Dato Riga 2, Colonna 2 \\
\end{longtable}

% \paragraph{Tabella - DOM}\label{tabella---dom}

% \includesvg{tabella.svg}

\subsubsection{PHP con PDO e SQLite}\label{php-con-pdo-e-sqlite}

Questo manuale mostra un semplice esempio di come utilizzare PHP con PDO
(PHP Data Objects) per interagire con un database SQLite e leggere i
dati degli ordini da una tabella, per poi visualizzarli in una tabella
HTML.

\paragraph{Prerequisiti}\label{prerequisiti}

\begin{itemize}
\tightlist
\item
  \textbf{Database SQLite:} Devi avere un database SQLite esistente
\end{itemize}

\paragraph{Passaggi}\label{passaggi}

\begin{enumerate}
\def\labelenumi{\arabic{enumi}.}
\item
  \textbf{Creazione del Database SQLite (se non esiste):}

  Se non hai già un database SQLite, devi crearlo. Supponiamo di avere
  un database nel file \texttt{mio\_database.db} con la tabella
  \texttt{ordini} e alcuni dati di esempio.
\item
  \textbf{Creazione del File PHP:}

  Crea un file PHP chiamato, ad esempio, \texttt{mostra\_ordini.php}.
\item
  \textbf{Scrittura del Codice PHP:}

\begin{Shaded}
\begin{Highlighting}[]
\OperatorTok{\textless{}!}\ConstantTok{DOCTYPE}\NormalTok{ html}\OperatorTok{\textgreater{}}
\OperatorTok{\textless{}}\NormalTok{html lang}\OperatorTok{=}\StringTok{"it"}\OperatorTok{\textgreater{}}
\OperatorTok{\textless{}}\NormalTok{head}\OperatorTok{\textgreater{}}
    \OperatorTok{\textless{}}\NormalTok{meta charset}\OperatorTok{=}\StringTok{"UTF{-}8"}\OperatorTok{\textgreater{}}
    \OperatorTok{\textless{}}\NormalTok{title}\OperatorTok{\textgreater{}}\NormalTok{Elenco Ordini}\OperatorTok{\textless{}/}\NormalTok{title}\OperatorTok{\textgreater{}}
    \OperatorTok{\textless{}}\NormalTok{style}\OperatorTok{\textgreater{}}
\NormalTok{        table \{}
\NormalTok{            border}\OperatorTok{{-}}\NormalTok{collapse}\OtherTok{:}\NormalTok{ collapse}\OtherTok{;}
\NormalTok{            width}\OtherTok{:} \DecValTok{80}\OperatorTok{\%}\OtherTok{;}
\NormalTok{            margin}\OtherTok{:} \ErrorTok{20}\NormalTok{px auto}\OtherTok{;}
\NormalTok{        \}}
\NormalTok{        th}\OtherTok{,}\NormalTok{ td \{}
\NormalTok{            border}\OtherTok{:} \ErrorTok{1}\NormalTok{px solid }\CommentTok{\#ddd;}
\NormalTok{            padding}\OtherTok{:} \ErrorTok{8}\NormalTok{px}\OtherTok{;}
\NormalTok{            text}\OperatorTok{{-}}\NormalTok{align}\OtherTok{:}\NormalTok{ left}\OtherTok{;}
\NormalTok{        \}}
\NormalTok{        th \{}
\NormalTok{            background}\OperatorTok{{-}}\NormalTok{color}\OtherTok{:} \CommentTok{\#f2f2f2;}
\NormalTok{        \}}
    \OperatorTok{\textless{}/}\NormalTok{style}\OperatorTok{\textgreater{}}
\OperatorTok{\textless{}/}\NormalTok{head}\OperatorTok{\textgreater{}}
\OperatorTok{\textless{}}\NormalTok{body}\OperatorTok{\textgreater{}}
    \OperatorTok{\textless{}}\NormalTok{h1}\OperatorTok{\textgreater{}}\NormalTok{Elenco Ordini}\OperatorTok{\textless{}/}\NormalTok{h1}\OperatorTok{\textgreater{}}

    \OperatorTok{\textless{}}\OtherTok{?}\NormalTok{php}
    \CommentTok{// Percorso al database SQLite}
    \VariableTok{$dbFile} \OperatorTok{=} \StringTok{\textquotesingle{}mio\_database.db\textquotesingle{}}\OtherTok{;}

    \ControlFlowTok{try}\NormalTok{ \{}
        \CommentTok{// Connessione al database SQLite usando PDO}
        \VariableTok{$pdo} \OperatorTok{=} \KeywordTok{new} \BuiltInTok{PDO}\NormalTok{(}\StringTok{"sqlite:"} \OperatorTok{.} \VariableTok{$dbFile}\NormalTok{)}\OtherTok{;}

        \CommentTok{// Imposta la modalità di errore PDO su eccezioni}
        \VariableTok{$pdo}\NormalTok{{-}\textgreater{}setAttribute(}\BuiltInTok{PDO}\NormalTok{::}\ConstantTok{ATTR\_ERRMODE}\OtherTok{,} \BuiltInTok{PDO}\NormalTok{::}\ConstantTok{ERRMODE\_EXCEPTION}\NormalTok{)}\OtherTok{;}

        \CommentTok{// Query per selezionare tutti gli ordini}
        \VariableTok{$sql} \OperatorTok{=} \StringTok{"SELECT id, data\_ordine, cliente, prodotto, quantita FROM ordini"}\OtherTok{;}
        \VariableTok{$stmt} \OperatorTok{=} \VariableTok{$pdo}\NormalTok{{-}\textgreater{}prepare(}\VariableTok{$sql}\NormalTok{)}\OtherTok{;}
        \VariableTok{$stmt}\NormalTok{{-}\textgreater{}execute()}\OtherTok{;}

        \CommentTok{// Recupera tutti i risultati come un array associativo}
        \VariableTok{$ordini} \OperatorTok{=} \VariableTok{$stmt}\NormalTok{{-}\textgreater{}fetchAll(}\BuiltInTok{PDO}\NormalTok{::}\ConstantTok{FETCH\_ASSOC}\NormalTok{)}\OtherTok{;}

        \CommentTok{// Verifica se ci sono ordini}
        \ControlFlowTok{if}\NormalTok{ (}\FunctionTok{count}\NormalTok{(}\VariableTok{$ordini}\NormalTok{) }\OperatorTok{\textgreater{}} \DecValTok{0}\NormalTok{) \{}
            \KeywordTok{echo} \StringTok{\textquotesingle{}\textless{}table\textgreater{}\textquotesingle{}}\OtherTok{;}
            \KeywordTok{echo} \StringTok{\textquotesingle{}\textless{}thead\textgreater{}\textquotesingle{}}\OtherTok{;}
            \KeywordTok{echo} \StringTok{\textquotesingle{}\textless{}tr\textgreater{}\textquotesingle{}}\OtherTok{;}
            \KeywordTok{echo} \StringTok{\textquotesingle{}\textless{}th\textgreater{}ID\textless{}/th\textgreater{}\textquotesingle{}}\OtherTok{;}
            \KeywordTok{echo} \StringTok{\textquotesingle{}\textless{}th\textgreater{}Data Ordine\textless{}/th\textgreater{}\textquotesingle{}}\OtherTok{;}
            \KeywordTok{echo} \StringTok{\textquotesingle{}\textless{}th\textgreater{}Cliente\textless{}/th\textgreater{}\textquotesingle{}}\OtherTok{;}
            \KeywordTok{echo} \StringTok{\textquotesingle{}\textless{}th\textgreater{}Prodotto\textless{}/th\textgreater{}\textquotesingle{}}\OtherTok{;}
            \KeywordTok{echo} \StringTok{\textquotesingle{}\textless{}th\textgreater{}Quantità\textless{}/th\textgreater{}\textquotesingle{}}\OtherTok{;}
            \KeywordTok{echo} \StringTok{\textquotesingle{}\textless{}/tr\textgreater{}\textquotesingle{}}\OtherTok{;}
            \KeywordTok{echo} \StringTok{\textquotesingle{}\textless{}/thead\textgreater{}\textquotesingle{}}\OtherTok{;}
            \KeywordTok{echo} \StringTok{\textquotesingle{}\textless{}tbody\textgreater{}\textquotesingle{}}\OtherTok{;}

            \CommentTok{// Itera attraverso gli ordini e crea le righe della tabella HTML}
            \ControlFlowTok{foreach}\NormalTok{ (}\VariableTok{$ordini} \KeywordTok{as} \VariableTok{$ordine}\NormalTok{) \{}
                \KeywordTok{echo} \StringTok{\textquotesingle{}\textless{}tr\textgreater{}\textquotesingle{}}\OtherTok{;}
                \KeywordTok{echo} \StringTok{\textquotesingle{}\textless{}td\textgreater{}\textquotesingle{}} \OperatorTok{.} \FunctionTok{htmlspecialchars}\NormalTok{(}\VariableTok{$ordine}\NormalTok{[}\StringTok{\textquotesingle{}id\textquotesingle{}}\NormalTok{]) }\OperatorTok{.} \StringTok{\textquotesingle{}\textless{}/td\textgreater{}\textquotesingle{}}\OtherTok{;}
                \KeywordTok{echo} \StringTok{\textquotesingle{}\textless{}td\textgreater{}\textquotesingle{}} \OperatorTok{.} \FunctionTok{htmlspecialchars}\NormalTok{(}\VariableTok{$ordine}\NormalTok{[}\StringTok{\textquotesingle{}data\_ordine\textquotesingle{}}\NormalTok{]) }\OperatorTok{.} \StringTok{\textquotesingle{}\textless{}/td\textgreater{}\textquotesingle{}}\OtherTok{;}
                \KeywordTok{echo} \StringTok{\textquotesingle{}\textless{}td\textgreater{}\textquotesingle{}} \OperatorTok{.} \FunctionTok{htmlspecialchars}\NormalTok{(}\VariableTok{$ordine}\NormalTok{[}\StringTok{\textquotesingle{}cliente\textquotesingle{}}\NormalTok{]) }\OperatorTok{.} \StringTok{\textquotesingle{}\textless{}/td\textgreater{}\textquotesingle{}}\OtherTok{;}
                \KeywordTok{echo} \StringTok{\textquotesingle{}\textless{}td\textgreater{}\textquotesingle{}} \OperatorTok{.} \FunctionTok{htmlspecialchars}\NormalTok{(}\VariableTok{$ordine}\NormalTok{[}\StringTok{\textquotesingle{}prodotto\textquotesingle{}}\NormalTok{]) }\OperatorTok{.} \StringTok{\textquotesingle{}\textless{}/td\textgreater{}\textquotesingle{}}\OtherTok{;}
                \KeywordTok{echo} \StringTok{\textquotesingle{}\textless{}td\textgreater{}\textquotesingle{}} \OperatorTok{.} \FunctionTok{htmlspecialchars}\NormalTok{(}\VariableTok{$ordine}\NormalTok{[}\StringTok{\textquotesingle{}quantita\textquotesingle{}}\NormalTok{]) }\OperatorTok{.} \StringTok{\textquotesingle{}\textless{}/td\textgreater{}\textquotesingle{}}\OtherTok{;}
                \KeywordTok{echo} \StringTok{\textquotesingle{}\textless{}/tr\textgreater{}\textquotesingle{}}\OtherTok{;}
\NormalTok{            \}}

            \KeywordTok{echo} \StringTok{\textquotesingle{}\textless{}/tbody\textgreater{}\textquotesingle{}}\OtherTok{;}
            \KeywordTok{echo} \StringTok{\textquotesingle{}\textless{}/table\textgreater{}\textquotesingle{}}\OtherTok{;}
\NormalTok{        \} }\ControlFlowTok{else}\NormalTok{ \{}
            \KeywordTok{echo} \StringTok{\textquotesingle{}\textless{}p\textgreater{}Nessun ordine trovato.\textless{}/p\textgreater{}\textquotesingle{}}\OtherTok{;}
\NormalTok{        \}}

        \CommentTok{// Chiudi la connessione}
        \VariableTok{$pdo} \OperatorTok{=} \KeywordTok{null}\OtherTok{;}

\NormalTok{    \} }\ControlFlowTok{catch}\NormalTok{ (}\BuiltInTok{PDOException} \VariableTok{$e}\NormalTok{) \{}
        \KeywordTok{echo} \StringTok{\textquotesingle{}\textless{}p\textgreater{}Errore di connessione o query: \textquotesingle{}} \OperatorTok{.} \VariableTok{$e}\NormalTok{{-}\textgreater{}getMessage() }\OperatorTok{.} \StringTok{\textquotesingle{}\textless{}/p\textgreater{}\textquotesingle{}}\OtherTok{;}
\NormalTok{    \}}
    \KeywordTok{?\textgreater{}}

\OperatorTok{\textless{}/}\NormalTok{body}\OperatorTok{\textgreater{}}
\OperatorTok{\textless{}/}\NormalTok{html}\OperatorTok{\textgreater{}}
\end{Highlighting}
\end{Shaded}
\end{enumerate}

\paragraph{Spiegazione del Codice PHP}\label{spiegazione-del-codice-php}

\begin{itemize}
\tightlist
\item
  \textbf{\texttt{\$dbFile\ =\ \textquotesingle{}mio\_database.db\textquotesingle{};}}:
  Definisce il percorso al file del database SQLite. Assicurati che
  questo percorso sia corretto.
\item
  \textbf{\texttt{try...catch}}: Blocca il codice che potrebbe generare
  eccezioni (come errori di connessione al database o errori nella
  query).
\item
  \textbf{\texttt{\$pdo\ =\ new\ PDO("sqlite:"\ .\ \$dbFile);}}: Crea un
  nuovo oggetto PDO per connettersi al database SQLite specificato. Il
  prefisso \texttt{"sqlite:"} indica il driver da utilizzare.
\item
  \textbf{\texttt{\$pdo-\textgreater{}setAttribute(PDO::ATTR\_ERRMODE,\ PDO::ERRMODE\_EXCEPTION);}}:
  Imposta la modalità di gestione degli errori di PDO per lanciare
  eccezioni in caso di problemi. Questo è utile per il debugging.
\item
  \textbf{\texttt{\$sql\ =\ "SELECT\ id,\ data\_ordine,\ cliente,\ prodotto,\ quantita\ FROM\ ordini";}}:
  Definisce la query SQL per selezionare tutte le colonne dalla tabella
  \texttt{ordini}.
\item
  \textbf{\texttt{\$stmt\ =\ \$pdo-\textgreater{}prepare(\$sql);}}:
  Prepara la query SQL per l'esecuzione. Anche se in questo caso la
  query è semplice e non contiene input esterni, la preparazione è una
  buona pratica per la sicurezza e l'efficienza, soprattutto con query
  più complesse.
\item
  \textbf{\texttt{\$stmt-\textgreater{}execute();}}: Esegue la query
  preparata.
\item
  \textbf{\texttt{\$ordini\ =\ \$stmt-\textgreater{}fetchAll(PDO::FETCH\_ASSOC);}}:
  Recupera tutte le righe risultanti dalla query come un array
  associativo. Ogni elemento dell'array \texttt{\$ordini} è un array con
  chiavi corrispondenti ai nomi delle colonne della tabella.
\item
  \textbf{\texttt{count(\$ordini)\ \textgreater{}\ 0}}: Verifica se sono
  stati trovati ordini nel database.
\item
  \textbf{Creazione della Tabella HTML}: Se ci sono ordini, il codice
  PHP genera dinamicamente una tabella HTML
  (\texttt{\textless{}table\textgreater{}},
  \texttt{\textless{}thead\textgreater{}},
  \texttt{\textless{}tbody\textgreater{}},
  \texttt{\textless{}tr\textgreater{}},
  \texttt{\textless{}th\textgreater{}},
  \texttt{\textless{}td\textgreater{}}) per visualizzare i dati.
\item
  \textbf{\texttt{htmlspecialchars()}}: Questa funzione viene utilizzata
  per rendere sicuri i dati visualizzati nella tabella HTML, prevenendo
  potenziali attacchi XSS (Cross-Site Scripting) convertendo caratteri
  speciali HTML nelle loro entità HTML.
\item
  \textbf{\texttt{\$pdo\ =\ null;}}: Chiude la connessione al database
  impostando l'oggetto PDO a \texttt{null}.
\item
  \textbf{\texttt{catch\ (PDOException\ \$e)}}: Cattura qualsiasi
  eccezione PDO che si verifica e visualizza un messaggio di errore.
\end{itemize}

\end{document}
