% -*- mode: latex; coding: utf8; TeX-master: itadinfo25-ricop.tex -*-
% !TeX root = itadinfo25-ricop.tex
% !TeX encoding = UTF-8 Unicode
% !Tex TS-program = LuaLaTeX
\documentclass[withtimes]{easychair}
\usepackage[utf8]{inputenc}
\usepackage[T1]{fontenc}
\usepackage[italian]{babel}
%\usepackage[autostyle,italian=guillemets]{csquotes}
\usepackage[backend=biber,style=numeric,hyperref]{biblatex}
\addbibresource{itadinfo25-ricop.bib}
\usepackage{doc}

%% Front Matter
%%
% Regular title as in the article class.
%
\pagestyle{empty}
\fontfamily{garamond}
\title{Un Percorso Sfidante tra Informatica e Matematica\\
Gamification e Metacognizione per l'apprendimento della Ricerca
Operativa}

% Authors are joined by \and. Their affiliations are given by \inst, which indexes
% into the list defined using \institute
%
\author{
Fabrizio Marinelli\inst{1}\thanks{Docente esperto che ha proposto il corso}
\and
Gionata Massi\inst{2}\thanks{Tutor}
}

% Institutes for affiliations are also joined by \and,
\institute{
  Dipartimento di Ingegneria dell'Informazione
  \\Universit\`a Politecnica delle Marche
  Ancona, Italia\\
  \email{fabrizio.marinelli@staff.univpm.it}
\and
   Istituto di Istruzione Superiore Savoia Benincasa\\
   Ancona, Italia\\
   \email{gionata.massi@savoiabenincasa.it}
 }

%  \authorrunning{} has to be set for the shorter version of the authors' names;
% otherwise a warning will be rendered in the running heads. When processed by
% EasyChair, this command is mandatory: a document without \authorrunning
% will be rejected by EasyChair

\authorrunning{Marinelli e Massi}

% \titlerunning{} has to be set to either the main title or its shorter
% version for the running heads. When processed by
% EasyChair, this command is mandatory: a document without \titlerunning
% will be rejected by EasyChair
\titlerunning{Un Percorso Sfidante tra Informatica e Matematica}

\begin{document}

\maketitle


\begin{abstract}
Il corso di introduzione alla \textbf{Ricerca Operativa} mira a
integrare le conoscenze multidisciplinari degli studenti, supportandoli
nelle scelte post-diploma. I suoi obiettivi includono lo sviluppo di
competenze di \textbf{problem solving} e \textbf{modellizzazione
quantitativa}, potenziando la \textbf{sinergia matematica-informatica} e
la motivazione verso le materie STEM.

Viene adottata una metodologia didattica innovativa basata su: (a) un
\textbf{approccio ``sfidante''} con problemi reali e complessi; (b) la
\textbf{costruzione di conoscenza di gruppo} tramite attività
collaborative; (c) la \textbf{gamification} con sfide tra gruppi e
individui; (d) la stimolazione del \textbf{processo metacognitivo}
attraverso scommesse sulla qualità delle soluzioni; (e) la segmentazione
delle attività per ottimizzare l'efficacia del tempo di spiegazione
teorica.

La sequenza didattica adatta gli argomenti alle competenze pregresse e
allo sviluppo di nuove, privilegiando una didattica attiva. Le
spiegazioni teoriche sono brevi e fornite solo quando lo studente ne
percepisce la necessità, in linea con un approccio ispirato al
\emph{Necessity Learning Design}~\cite{Sbaraglia}. In questo contesto, il meccanismo
della ``scommessa'' non solo attiva la metacognizione ma stimola anche
la ricerca di una dimostrazione rigorosa della qualità del
proprio lavoro.

L'impatto sul coinvolgimento degli studenti è stato elevato, con
significativi risultati nell'attivazione dei meccanismi di problem
solving e modellizzazione. Basandosi sui questionari di gradimento, il
percorso didattico ha pienamente realizzato gli obiettivi prefissati.
\end{abstract}

\section{Introduzione}\label{sect:introduzione}

Questo articolo presenta un'esperienza didattica innovativa sviluppata
nell'ambito del DM 65/2023{[}\^{}1{]}, finalizzata al
\textbf{potenziamento delle competenze STEM{[}\^{}2{]}, digitali e di
innovazione} tramite un corso di introduzione alla \textbf{Ricerca
Operativa}. Il progetto ha coinvolto 16 studenti di una classe V
dell'articolazione Sistemi Informativi Aziendali dell'Istituto Tecnico
Economico ``Grazioso Benincasa'' di Ancona{[}\^{}3{]}.

Nelle programmazioni curricolari di tale indirizzo, le tematiche della
Ricerca Operativa sono spesso trattate in modo non organico, frammentate
tra matematica, informatica ed economia aziendale. Il corso proposto, in
linea con gli obiettivi del DM 65/2023, mira a:

\begin{itemize}
\tightlist
\item
  Integrare conoscenze e abilità multidisciplinari per affrontare e
  risolvere problemi quantitativi complessi.
\item
  Applicare principi matematici e algoritmi a problemi decisionali non
  banali.
\item
  Sviluppare competenze di problem solving scientifico e pensiero
  critico.
\item
  Accrescere la motivazione verso lo studio delle discipline STEM, in
  particolare matematica e informatica.
\item
  Costruire un sapere multidisciplinare che metta in nuova luce
  l'importanza di discipline tradizionali.
\end{itemize}

Il percorso didattico si è articolato in \textbf{cinque incontri di due
ore ciascuno}, nei quali sono state proposte 11 domande individuali e 5
sfide di gruppo.

\section{Approccio Didattico}\label{approccio-didattico}

Il corso è stato concepito come un \textbf{gioco} per massimizzare il
coinvolgimento degli studenti, con le seguenti regole:

\begin{itemize}
\tightlist
\item
  Ogni partecipante riceve una dotazione iniziale di 1000 ``dobloni''.
\item
  I partecipanti sono raggruppati in squadre.
\item
  I dobloni si guadagnano rispondendo correttamente a domande
  individuali.
\item
  Le squadre partecipano a gare, scommettendo da 10 a 200 dobloni sulla
  propria performance.
\end{itemize}

Le attività di gruppo si basano sulla \textbf{sfida cooperativa}. Viene
presentato un problema decisionale accattivante e realistico, prima di
ogni spiegazione teorica formale. I gruppi, in competizione, sono
invitati a determinare soluzioni al problema sotto forma di numeri,
modelli o algoritmi.

Al termine del tempo prestabilito, i gruppi consegnano la loro soluzione
e scommettono una quantità di dobloni sulla qualità della propria
proposta rispetto a quelle degli altri gruppi. Segue la fase di
attribuzione dei punteggi, durante la quale il formatore mostra
pubblicamente le soluzioni e valuta le performance. Al termine della
valutazione, uno studente per gruppo esplicita le strategie utilizzate.

L'ultima fase è la \textbf{formalizzazione guidata}, dove il formatore
illustra la teoria. Questo avviene nel momento in cui la conoscenza
formale diventa ``necessaria'' per gli studenti, per tradurre il
problema in variabili, funzione obiettivo e vincoli, o per
visualizzare/calcolare lo spazio delle soluzioni e definire strategie o
algoritmi. L'induzione nell'esigenza di un nuovo modello o costrutto è
simile al \emph{Necessity Learning Design} (NLD), differenziandosi
leggermente in quanto gli studenti potrebbero aver già trovato una
soluzione empirica al problema.

L'approccio è affine al \emph{Problem-Based Learning} (PBL) ma ne
potenzia il coinvolgimento attivo attraverso elementi di
\textbf{gamification}:

\begin{itemize}
\tightlist
\item
  Vengono proposte situazioni complesse del mondo reale, le quali, a
  differenza di quanto rigidamente prescritto dal PBL, possiedono una
  soluzione quantitativa ``esatta'' o ottimale.
\item
  Gli studenti lavorano in gruppo, confrontandosi con il problema,
  sviluppando possibili modelli e algoritmi risolutivi.
\item
  Gli studenti costruiscono una conoscenza di gruppo attraverso
  l'apprendimento autodiretto, induttivo e cooperativo.
\item
  Viene sviluppata e potenziata la competenza di \emph{problem solving}.
\end{itemize}

L'introduzione della scommessa ha la duplice finalità di attivare il
processo metacognitivo e di indurre l'esigenza di \textbf{dimostrare
formalmente la bontà della soluzione proposta}.

\section{Contenuti}\label{contenuti}

Il percorso didattico ha affrontato diverse tipologie di problemi di
Ricerca Operativa.

Il primo problema proposto è stato quello dello \textbf{zaino (0-1
knapsack problem)}. La sfida individuale iniziale ha richiesto agli
studenti di selezionare, tra quattro oggetti con peso e valore
specifici, quelli da inserire in un bagaglio a mano per massimizzare il
valore complessivo senza superare un limite di peso. Successivamente, è
stata mostrata l'enumerazione di tutte le soluzioni possibili. Il
problema è stato contestualizzato a scenari reali come il caricamento di
container o veicoli. La sfida di gruppo ha poi richiesto di caricare un
furgone scegliendo tra 50 oggetti, massimizzando il valore trasportato.
Al termine, sono state analizzate le strategie algoritmiche dei gruppi,
focalizzando la discussione sulla necessità di determinare se la
soluzione trovata fosse la migliore possibile.

Successivamente, si è affrontato il problema della
\textbf{determinazione del rettangolo di area massima dato il
perimetro}. La sfida individuale ha previsto l'uso di un foglio di
calcolo per determinare le lunghezze dei lati. La soluzione analitica è
stata poi confrontata con procedure di calcolo basate su simulazioni
numeriche, inducendo la necessità di un modello matematico per
dimostrare formalmente l'ottimalità. È stato introdotto un linguaggio
per esprimere problemi di programmazione matematica (es. AMPL). La
seconda sfida di gruppo ha richiesto di descrivere un problema
ambientato nel \emph{vecchio west} nei termini della programmazione
matematica, definendo variabili di decisione, funzione obiettivo e
vincoli.

Si è poi ritornati al problema dello zaino per un processo metacognitivo
sulle modalità risolutive. Sono state proposte sfide individuali sulla
determinazione del numero di sotto-insiemi e sulla definizione di
variabili e vincoli per il problema dello zaino. La discussione ha
condotto alla comprensione dell'\textbf{intrattenibilità computazionale}
del problema dello zaino con molti oggetti tramite ricerca esaustiva
(\(2^{50}\) soluzioni). Dal problema reale, si sono astratte le entità
del modello: variabili, funzione obiettivo e vincoli, richiamando
concetti noti come variabili booleane, array e codifiche binarie, e
mostrando esempi reali di problemi con complessità esponenziale.

Dai problemi combinatori si è passati alla \textbf{programmazione
lineare} con una sfida di gruppo su un problema di mix ottimo di
produzione (due variabili) nel settore dolciario. Lo sviluppo della
conoscenza è proseguito con sfide individuali sulla determinazione di
variabili di decisione, vincoli e funzione obiettivo, e sulla ricerca
della soluzione ottima. La discussione delle strategie ha fatto
emergere, seppur senza conoscenza formale del metodo, principi analoghi
alle strategie di scelta delle variabili nel metodo del simplesso e al
test di ammissibilità. Attraverso un processo di elicitazione, è stato
costruito il modello matematico completo del problema, quindi illustrato
il metodo del simplesso in forma grafica. La gara di gruppo successiva
ha previsto l'applicazione del metodo grafico del simplesso ad
un'istanza di programma lineare, attivando conoscenze pregresse su
rette, derivate, gradienti e curve di livello.

Infine, è stata proposta una sfida di gruppo sul \textbf{problema della
dieta} con più di due alimenti. Anche per questo problema, tramite sfide
individuali, si sono identificate le variabili di decisione, i vincoli e
la funzione obiettivo. È seguita un'illustrazione sui concetti di spazi
a dimensioni maggiori di tre per fornire una base di conoscenza per la
risoluzione di problemi di programmazione lineare più complessi.

\section{Risultati}\label{risultati}

Gli studenti hanno mostrato un \textbf{coinvolgimento e una motivazione
mediamente superiori} rispetto alle lezioni tradizionali. Su 16
studenti, 15 hanno partecipato ad almeno 7 ore del corso, con l'unica
assenza prolungata dovuta a problemi di salute.

La maggioranza degli studenti ha manifestato un \textbf{accresciuto
interesse per la matematica e l'informatica}, percepite in una nuova
ottica. Questa osservazione è corroborata dai questionari di gradimento
anonimi, dove 10 studenti su 15 hanno affermato che il formatore è stato
``capace di suscitare interesse e coinvolgere'', 9 hanno ritenuto il
corso utile per le informazioni fornite, 8 per le abilità/capacità
operative sviluppate e 7 per l'interesse complessivo suscitato.

Lo sviluppo delle competenze di problem solving e modellizzazione è
stato evidente negli esiti delle sfide individuali e di gruppo, dove gli
studenti hanno dimostrato capacità di analizzare un problema,
identificare le entità del modello e tentare una formalizzazione.
L'elemento della scommessa ha stimolato una profonda \textbf{riflessione
sulla validità delle proprie strategie e soluzioni}, sebbene non siano
state osservate intense discussioni e argomentazioni \emph{all'interno}
dei gruppi riguardo le scommesse stesse, quanto piuttosto sulla
soluzione del problema.

La \textbf{spiegazione formale è risultata più significativa e meglio
compresa} dopo che gli studenti si erano confrontati attivamente con il
problema. Gli studenti hanno colto il ruolo complementare di matematica
e informatica, comprendendo meglio le distinzioni tra descrizioni
dichiarative (es. linguaggio matematico, SQL) e linguaggi procedurali.

Il lavoro di gruppo si è svolto, con l'eccezione di un singolo studente
particolarmente demotivato, in modo molto partecipe. Anche studenti
solitamente meno coinvolti hanno contribuito attivamente alla
risoluzione delle sfide.

\emph{(Considerazione: Inserire qui un'immagine sintetica o un grafico
sui risultati del questionario di gradimento potrebbe essere molto
efficace, se lo spazio lo consente.)}

Un limite riscontrato è stato la \textbf{ristrettezza dei tempi}: 10 ore
complessive potrebbero essere state insufficienti per approfondire tutti
gli aspetti, suggerendo che 15 ore sarebbero state più adeguate.

\section{Conclusioni e Prospettive
Future}\label{conclusioni-e-prospettive-future}

L'approccio didattico descritto si è rivelato \textbf{efficace nel
raggiungere gli obiettivi formativi del DM 65/2023}. L'approccio attivo
stimola gli studenti nel processo di costruzione di una conoscenza più
profonda. La gamification aumenta la motivazione e la collaborazione,
mentre il meccanismo della scommessa stimola la metacognizione e la
ricerca autonoma di validazione. La spiegazione teorica, presentata solo
dopo il tentativo empirico degli studenti, risulta più interessante e
l'alternanza tra fasi attive e momenti di ascolto della spiegazione
ottimizza l'efficacia della lezione.

Questo modello didattico può essere applicato con successo ad altri
argomenti dell'informatica o della matematica, anche in contesti
educativi diversi. In particolare, si adatta bene al Liceo Scientifico
(indirizzo Scienze Applicate). Per annualità differenti o altri ordini
di scuola, i contenuti dovrebbero essere adattati alle conoscenze
pregresse e alle capacità di astrazione tipiche della fascia d'età.

In prospettiva futura, si potrebbe sviluppare una versione online del
sistema di assegnazione dei punteggi come \emph{proof of concept} per
una piattaforma didattica. Sarebbe inoltre prezioso realizzare una
seconda edizione del corso con un \textbf{monitoraggio più sistematico e
scientifico} del conseguimento delle competenze e dell'efficacia
didattica, per raccogliere dati più robusti a supporto delle
osservazioni qualitative iniziali.

\section*{Ringraziamenti}
\label{sect:ringraziamenti}

Gli autori ringraziano le prof.sse Annamaria Rossi e Silvana Sabracone, rispettivamente insegnanti di Matematica ed Economia Aziendale nella classe target, per aver concesso parte delle loro ore al fine di realizzare il progetto e per aver assistito a parte lezioni del corso. I loro feedback contribuiranno a migliorare il progetto. 

\label{sect:bib}
%\bibliographystyle{plain}
%\bibliography{itadinfo25-ricop}
\printbibliography

\end{document}